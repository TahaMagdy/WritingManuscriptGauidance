\section{Datasets: Acquisition, Encoding, and Repository}\label{sec:datasets}
Sec.~\ref{sec:arabic_english_dataset} explains how the Arabic and English datasets were scraped from
different non-technical web sources; and hence needed a lot of cleaning and structuring. For future research on
these datasets, and probably for collecting more poem datasets, we launched the data repository
``Poem Comprehensive Dataset (PCD)''~\citep{Yousef2018PoemComprehensiveDataset} that is publicly
available for the whole community. The datasets on this repository are in their final clean formats
and ready for computational purposes. Sec.~\ref{sec:data-encoding} explains the data encoding at the
character level before feeding to the RNN\@.

\subsection{Arabic and English Datasets Acquisition}\label{sec:arabic_english_dataset}
\begin{figure}[!tb]
  \centering
  \begin{tikzpicture}[scale=0.8]
    \begin{axis}[
  symbolic x coords={Taweel,
    Kamel,
    Baseet,
    Khafeef,
    Wafeer,
    Rigz,
    Raml,
    Motakarib,
    Sar'e,
    Monsareh,
    Mogtath,
    Madeed,
    Hazg,
    Motadarik,
    Moktadib,
    Modar'e
  },
  xtick=data,
  every axis y label/.style= {at={( 0.15, 1.07)}, anchor=north},
  ylabel style={font=\footnotesize},
  xticklabel style = {font=\footnotesize},
  ylabel={Verses},
  height=8cm,
  x=0.4cm,
  x tick label style={rotate=90, anchor=east},
  enlarge y limits=0.07,
  name=left plot,%
  title=(a) \textit{Arabic Dataset},%
  title style={at={(0.5,-.4)}}%
  ]
  \addplot[ybar,color=black,mark=*, only marks,
  point meta=explicit symbolic] coordinates {
    (Taweel, 416428)
    (Kamel,  370116)
    (Baseet, 244583)
    (Khafeef,     157880)
    (Wafeer,     143148)
    (Rigz,     119286)
    (Raml,     79560)
    (Motakarib,     63613)
    (Sar'e,     59370)
    (Monsareh,     28768)
    (Mogtath,     18062)
    (Madeed,     7808)
    (Hazg,     7468)
    (Motadarik,     5144)
    (Moktadib,     799 )
    (Modar'e,     288 )
  };

  \draw[loosely dotted] (axis cs:Taweel, 0) -- (axis cs:Taweel, 416428);
  \draw[loosely dotted] (axis cs:Kamel,  0) -- (axis cs:Kamel,  370116);
  \draw[loosely dotted] (axis cs:Baseet, 0) -- (axis cs:Baseet, 244583);
  \draw[loosely dotted] (axis cs:Khafeef,0) -- (axis cs:Khafeef,157880);
  \draw[loosely dotted] (axis cs:Wafeer, 0) -- (axis cs:Wafeer, 143148);
  \draw[loosely dotted] (axis cs:Rigz,   0) -- (axis cs:Rigz,   119286);
  \draw[loosely dotted] (axis cs:Raml,   0) -- (axis cs:Raml,   79560);
  \draw[loosely dotted] (axis cs:Motakarib, 0) -- (axis cs:Motakarib, 63613);
  \draw[loosely dotted] (axis cs:Sar'e, 0)   -- (axis cs:Sar'e, 59370);
  \draw[loosely dotted] (axis cs:Monsareh, 0) -- (axis cs:Monsareh,28768);
  \draw[loosely dotted] (axis cs:Mogtath, 0) -- (axis cs:Mogtath, 18062);
  \draw[loosely dotted] (axis cs:Madeed,  0) -- (axis cs:Madeed,  7808);
  \draw[loosely dotted] (axis cs:Hazg,    0) -- (axis cs:Hazg,    7468);
  \draw[loosely dotted] (axis cs:Motadarik,0) -- (axis cs:Motadarik, 5144);
  \draw[loosely dotted] (axis cs:Moktadib, 0) -- (axis cs:Moktadib,  799 );
  \draw[loosely dotted] (axis cs:Modar'e,  0) -- (axis cs:Modar'e,   288 );

\end{axis}

    \begin{axis}[
    ylabel={Verses},
        symbolic x coords={Iambic, Trochee, Anapaestic, Dactyl},
        xtick=data,
        every axis y label/.style= {at={( 0.5, 1.07)}, anchor=north},
        ylabel style={font=\footnotesize},
        xticklabel style = {font=\footnotesize},
        height=8cm,
        x=0.5cm,
        enlarge x limits=0.3,
        enlarge y limits=0.07,
        nodes near coords,
        x tick label style={rotate=90, anchor=east},
        y tick label style={scaled ticks=base 10:-3},
        ymin=0,
        at={($(left plot.south east)+(0.7cm,0)$)},%
        title=(b) \textit{English Dataset},%
        title style={at={(0.5,-.4)}}%
      ]

  \addplot[mark=*, only marks,
    point meta=explicit symbolic] coordinates {
    (Trochee,    5418)
    (Anapaestic, 5378)
    (Dactyl,     1397)
    (Iambic,     5550)
};

    \draw[loosely dotted] (axis cs:Trochee, 0) -- (axis cs:Trochee, 5418);
    \draw[loosely dotted] (axis cs:Anapaestic, 0) -- (axis cs:Anapaestic, 5378);
    \draw[loosely dotted] (axis cs:Dactyl, 0) -- (axis cs:Dactyl, 1397);
    \draw[loosely dotted] (axis cs:Iambic, 0) -- (axis cs:Iambic, 5550);
\end{axis}%

  \end{tikzpicture}%
  \caption{\footnotesize Class size (number of verses), of both Arabic and English datasets, ordered
    descendingly on $y$ axis vs.\ corresponding meter name on $x$
    axis.}\label{fig:footn-footn-class}
\end{figure}
We have scrapped the Arabic dataset from two big poetry websites~\citep{diwan,
  PoetryEncyclopedia2016}; then both are merged into one large dataset. The total number of verses
is 1,862,046. Each verse is labeled by its meter, the poet who authored it, and the age it belongs
to. Overall, there are 22 meters, 3701 poets and 11 ages: Pre-Islamic, Islamic, Umayyad, Mamluk,
Abbasid, Ayyubid, Ottoman, Andalusian, the era between Umayyad and Abbasid, Fatimid, and modern. We are
only interested in the 16 classic meters which are attributed to \textit{Al-Farahidi}. These meters
comprise the majority of the dataset with a total number of 1,722,321
verses. Figure~\ref{fig:footn-footn-class}-a is an ordered bar chart of the number of verses per
meter. It is important to mention that the state of verse diacritic is inconsistent; a verse can
carry full, partial, or no diacritics. This should affect the accuracy results as discussed in
Sec.~\ref{sec:discussion}.

The English dataset is scraped from many different web
resources~\cite{HuberEighteenthCenturyPoetryArchive}. It consists of 199,002 verses; each of them is
labeled with one of the four meters: \textit{Iambic}, \textit{Trochee}, \textit{Dactyl} and,
\textit{Anapaestic}. Since the \textit{Iambic} class dominates the dataset with 186,809 verses, we
downsampled it to 5550 verses to keep classes almost balanced. Figure~\ref{fig:footn-footn-class}-b
is an ordered bar chart of the number of verses per meter.

\bigskip

For both Arabic and English datasets, data cleaning was tedious but necessary step before direct
computational use. The poem contained non-alphabetical characters, unnecessary in-text white spaces,
redundant glyphs, and inconsistent diacritics. E.g., the Arabic dataset in many places contained two
consecutive \textit{harakah} on the same letter or a \textit{harakah} after a white space. In
addition, as a pre-encoding step, we have factored a letter having either \textit{shaddah} or
\textit{tanween} into two letters, as explained in Sec.~\ref{sec:arabic-language}. This step
shortens the encoding vector and saves more memory as explained in the next section. Each of the
Arabic and English datasets, after merging and cleaning, is labeled and structured in its final
format that is made publicly available \citep{Yousef2018PoemComprehensiveDataset} as introduced
above.


\subsection{Data Encoding}\label{sec:data-encoding}
\begin{figure*}[!tb]
  \centering
  \begin{tikzpicture}
    \node at (-12,0){\begin{tikzpicture}   
  % Rectangles
  % the space between each rectangle
  \def \x {0.57} 
  \def \s {1} % I'll use it to shift to right
  % A list of letters <3
  \newcommand\letters{{"ا","بَ","حَ", "رْ", "مَ"}}

  % Rectangle
  \def \length {2.95}
  \foreach \j in {0,...,4} 
  \draw [rounded corners=1pt] (\s+\x*\j, 0) rectangle (\s+0.5+\x*\j, \length) 
  node[above, xshift=-0.25cm]
  {\textarabic{\pgfmathparse{\letters[\j]}\pgfmathresult}};

  % Above Annotation
  % Cell line
  % 0.25*10+0.09 = 2.59
  \def \d {3.9 - 0.15}
  \draw (\s+0.25-0.007, 3.9 ) -- (\s+2.53+0.007, 3.9 ); % the main bar
  \draw (\s+0.25, 3.9 ) -- (\s+0.25, \d ); % sub bar #ا
  \draw (\s+0.25*3+0.03, 3.9 ) -- (\s+0.25*3+0.03, \d ); % sub bar #ب
  \draw (\s+0.25*5+0.15, 3.9+0.15 ) -- (\s+0.25*5+0.15, \d ) 
  node[above, yshift=0.3cm]{\textarabic{مَرْحَبَا}}; % sub bar #ح
  \draw (\s+0.25*7+0.2, 3.9 ) -- (\s+0.25*7+0.2, \d ); % sub bar #ر
  \draw (\s+2.53, 3.9 )  -- (\s+2.53, \d );  % sub-bar #م

  % adding numbers, vector ا
  \node at (\s+0.25, \length - 0.25)    {1}; % the top number; using -
  \node at (\s+0.25, \length - 0.6)     {0};
  \node at (\s+0.25, \length - 0.95)    {0};
  \node at (\s+0.25, \length - 1.3)     {0};
  \node at (\s+0.25,  1.1)     {\vdots};
  % \node at (\s+0.25, \length - 1.65)    {0};
  % \node at (\s+0.25, \length - 2)       {0};
  % \node at (\s+0.25, \length - 2.35)    {0};
  \node at (\s+0.25, \length - 2.7)     {0};

  % distance between vectors elements
  \def \d {0.07}
  % adding numbers, vector ب
  \node at (\s+0.25*3+\d, \length - 0.25)    {0}; % the top number; using -
  \node at (\s+0.25*3+\d, \length - 0.6)     {\vdots};
  % \node at (\s+0.25*3+\d, \length - 0.95)    {0};
  \node at (\s+0.25*3+\d, \length - 1.18)     {1};
  \node at (\s+0.25*3+\d, 1.28) {\vdots};
  \node at (\s+0.25*3+\d, \length - 2.35)    {0};
  \node at (\s+0.25*3+\d, 0.25) {0}; % The bottom  number


  % adding numbers, vector ح
  \node at (\s+0.25*5+\d*2, \length - 0.25)    {0}; % the top number; using -
  \node at (\s+0.25*5+\d*2, \length - 0.6)     {\vdots};
  \node at (\s+0.25*5+\d*2, \length - 1.35)     {1};
  \node at (\s+0.25*5+\d*2, 1.18) {\vdots};
  \node at (\s+0.25*5+\d*2, \length - 2.35)    {0};
  \node at (\s+0.25*5+\d*2, 0.25) {0}; % The bottom  number

  % adding numbers, vector ر
  \node at (\s+0.25*7+\d*3, \length - 0.25)    {0}; % the top number; using -
  \node at (\s+0.25*7+\d*3, \length - 0.6)     {0};
  \node at (\s+0.25*7+\d*3, \length - 0.95)     {0};
  \node at (\s+0.25*7+\d*3, \length - 1.33)     {\vdots};
  \node at (\s+0.25*7+\d*3, 1.07) {1};
  \node at (\s+0.25*7+\d*3, \length - 2.2)    {\vdots};
  \node at (\s+0.25*7+\d*3, 0.25) {0}; % The bottom  number

  % adding numbers, vector م
  \node at (\s+0.25*9+\d*4, 3 - 0.25) {0}; % the top number; using -
  \node at (\s+0.25*9+\d*4, 3 - 0.69) {0};
  \node at (\s+0.25*9+\d*4, 0.25) {0};
  \node at (\s+0.25*9+\d*4, 0.25*4 - 0.1) {\vdots};
  \node at (\s+0.25*9+\d*4, 0.9 + 0.4) {1};
  \node at (\s+0.25*9+\d*4, 1.3 + 0.6) {\vdots};

  % Vector size annotations.
  % figures starts at x = 1
  \def \d {0.6} % Annotation x=0.4
  \draw[arrows=->] (\d, 1.7) -- (\d, 2.9); % up-line
  \draw[arrows=<-] (\d, 0.1) -- (\d, 1.1) node[above, yshift=0.07cm]
  {\small{181}};
  % bottom-line

\end{tikzpicture}%
};
    \node at (-7,0){\begin{tikzpicture}[x=1cm,y=1cm]

  % Rectangles
  % the space between each rectangular
  \def \x {0.57}
  \def \s {1} % I'll use it to shift to right
  % A list of letters <3
  \newcommand\letters{{"ا","بَ","حَ", "رْ", "مَ"}}

  % \draw[step=0.2, gray, very thin, red] (1,-0.5) grid (4,4);

  % Rectangles
  \def \length {2.95}
  \foreach \j in {0,...,4}
  \draw[rounded corners=1pt] (\s+\x*\j, 0) rectangle (\s+0.5+\x*\j, \length) 
  node[above, xshift=-0.25cm]
  {\textarabic{\pgfmathparse{\letters[\j]}\pgfmathresult}};

  % Above Annotation
  % Cell line
  % 0.25*10+0.09 = 2.59
  \def \d {3.9 - 0.15}
  \def \z {0.2}
  \draw (\s+0.25-0.007, 3.9 ) -- (\s+2.53+0.007, 3.9 ); % the main bar
  \draw (\s+0.25, 3.9 ) -- (\s+0.25, \d ); % sub bar #ا
  \draw (\s+0.25*3+0.03, 3.9 ) -- (\s+0.25*3+0.03, \d ); % sub bar #ب
  \draw (\s+0.25*5+0.15, 3.9+0.15 ) -- (\s+0.25*5+0.15, \d ) 
  node[above, yshift=0.3cm]{\textarabic{مَرْحَبَا}}; % sub bar #ح
  \draw (\s+0.25*7+0.2, 3.9 ) -- (\s+0.25*7+0.2, \d ); % sub bar #ر
  \draw (\s+2.53, 3.9 )  -- (\s+2.53, \d );  % sub-bar #م

  % adding numbers, vector ا
  \node at (\s+0.25, \length - 0.25)    {0}; % the top number; using -
  \node at (\s+0.25, \length - 0.6)     {0};
  \node at (\s+0.25, \length - 0.95)    {0};
  \node at (\s+0.25, \length - 1.3)     {0};
  \node at (\s+0.25, \length - 1.65)    {0};
  \node at (\s+0.25, \length - 2)       {0};
  \node at (\s+0.25, \length - 2.35)    {0};
  \node at (\s+0.25, \length - 2.7)     {1};

  % distance between vectors elements
  \def \d {0.07}
  % adding numbers, vector ب
  \node at (\s+0.25*3+\d*1, \length - 0.25)    {0}; % the top number; using -
  \node at (\s+0.25*3+\d*1, \length - 0.6)     {0};
  \node at (\s+0.25*3+\d*1, \length - 0.95)    {1};
  \node at (\s+0.25*3+\d*1, \length - 1.3)     {0};
  \node at (\s+0.25*3+\d*1, \length - 1.65)    {0};
  \node at (\s+0.25*3+\d*1, \length - 2)       {1};
  \node at (\s+0.25*3+\d*1, \length - 2.35)    {1};
  \node at (\s+0.25*3+\d*1, \length - 2.7)     {1};


  % adding numbers, vector ح
  \node at (\s+0.25*5+\d*2, \length - 0.25)    {0}; % the top number; using -
  \node at (\s+0.25*5+\d*2, \length - 0.6)     {0};
  \node at (\s+0.25*5+\d*2, \length - 0.95)    {1};
  \node at (\s+0.25*5+\d*2, \length - 1.3)     {0};
  \node at (\s+0.25*5+\d*2, \length - 1.65)    {1};
  \node at (\s+0.25*5+\d*2, \length - 2)       {0};
  \node at (\s+0.25*5+\d*2, \length - 2.35)    {1};
  \node at (\s+0.25*5+\d*2, \length - 2.7)     {1};

  % adding numbers, vector ر
  \node at (\s+0.25*7+\d*3, \length - 0.25)    {1}; % the top number; using -
  \node at (\s+0.25*7+\d*3, \length - 0.6)     {0};
  \node at (\s+0.25*7+\d*3, \length - 0.95)    {0};
  \node at (\s+0.25*7+\d*3, \length - 1.3)     {1};
  \node at (\s+0.25*7+\d*3, \length - 1.65)    {1};
  \node at (\s+0.25*7+\d*3, \length - 2)       {0};
  \node at (\s+0.25*7+\d*3, \length - 2.35)    {1};
  \node at (\s+0.25*7+\d*3, \length - 2.7)     {1};

  % adding numbers, vector م
  \node at (\s+0.25*9+\d*4, \length - 0.25)    {0}; % the top number; using -
  \node at (\s+0.25*9+\d*4, \length - 0.6)     {0};
  \node at (\s+0.25*9+\d*4, \length - 0.95)    {1};
  \node at (\s+0.25*9+\d*4, \length - 1.3)     {1};
  \node at (\s+0.25*9+\d*4, \length - 1.65)    {1};
  \node at (\s+0.25*9+\d*4, \length - 2)       {1};
  \node at (\s+0.25*9+\d*4, \length - 2.35)    {0};
  \node at (\s+0.25*9+\d*4, \length - 2.7)     {1};

  % Vector size annotations.
  % figures starts at x = 1
  \def \vColumn {0.6} % Annotation x=0.4
  \draw[arrows=->] (\vColumn, 1.7) -- (\vColumn, 2.9); % up-line
  \draw[arrows=<-] (\vColumn, 0.1) -- (\vColumn, 1.1) node[above, yshift=0.07cm] {\small{8}};

\end{tikzpicture}
};
    \node at (0,0){% Defining a new box
\newsavebox{\columnVector}
% Populating the box
\savebox{\columnVector}{$\left[\begin{smallmatrix}k\\m\end{smallmatrix}\right]$}

\begin{tikzpicture}

  % The grid
  % \draw[step=0.5, gray!40, very thin] (0,0) grid (6,4);

  \def \recWidthA {0.5}
  \def \yStartA   {1.5}
  \def \yEndA     {3}
  \def \xStartA   {1.5}
  \def \xEndA     {0.5}
  \def \xStartB   {0}
  \def \yStartB   {1}
  \def \xEndB     {2}
  \def \yEndB     {3.5}
  \def \half      {\yEndA/2 - \yStartA/2}
  \def \xStartC   {3 +0.5}
  \def \yStartC   {0.5 -.1}
  \def \xEndC     {3.5 +0.5}
  \def \yEndC     {4 +.1 }
  \def \shiftMargin {0.3}
  \newcommand\numbers{{0.25, 0.6, 0.95, 1.3, 1.65, 2, 2.35, 2.7}}

  % Rectangle A
  \draw[rounded corners=2pt] (\xStartA, \yStartA) rectangle (\xStartA + \recWidthA, \yEndA)
  node [above, xshift=-.25cm] {\textarabic{◌َ}};
  % Populating the rectangle A
  \node at (0.25 + \xStartA, 3 -\numbers[0]) {1};
  \foreach \j in {1,...,3}
  \node at (0.25 + \xStartA, \yEndA -\numbers[\j]) {0};

  % Annotation A
  \node at (\xStartA-\shiftMargin, \yStartA + \half) {\small 4};
  \draw[arrows=-angle 90, very thin] (\xStartA-\shiftMargin, \yStartA + \half +.3) -- (\xStartA-\shiftMargin, \yEndA -0.1);
  \draw[arrows=angle 90-, very thin] (\xStartA-\shiftMargin, \yStartA + .1) -- (\xStartA-\shiftMargin, \yStartA + \half -.3);

  % Rectangle B
  \draw[rounded corners=2pt] (\xStartB +0.5, \yStartB) rectangle (\yStartB +0.5 -1.5, \yEndB)
  node[above, xshift=0.25cm] {\textarabic{ب}};
  \node at (\xStartB +0.25, \yEndB -\numbers[0]) {0};
  \node at (\xStartB +0.25, \yEndB -\numbers[1]) {1};
  \node at (\xStartB +0.25, \yEndB -\numbers[2]) {0};
  \node at (\xStartB +0.25, \yEndB -\numbers[3]) {0};
  \node at (\xStartB +0.25, \yEndB -\numbers[4] ) {\vdots};
  \node at (\xStartB +0.25, \yStartB +\numbers[0]) {0};

  % Annotation B
  \node at (-\shiftMargin, \yStartA + \half) {\small 37};
  \draw[arrows=-angle 90, very thin] (-\shiftMargin, \yStartA + \half +.3) -- (-\shiftMargin, \yEndB -0.1);
  \draw[arrows=angle 90-, very thin] (-\shiftMargin, \yStartB + .1) -- (-\shiftMargin, \yStartA + \half -.3);

  \node at (0.8, \yStartA +\half) {$+$};
  \node at (2.5 + .02, \yStartA +\half) {$=$};

  % Rectangle C
  \draw[rounded corners=2pt]  (\xStartC, \yStartC) rectangle (\xEndC, \yEndC)
  node[above, xshift=-0.25cm] {\textarabic{بَ}};
  ;
  \node at (\xStartC +0.25, \yEndC -\numbers[0]) {1};
  \node at (\xStartC +0.25, \yEndC -\numbers[1]) {0};
  \node at (\xStartC +0.25, \yEndC -\numbers[2]) {0};
  \node at (\xStartC +0.25, \yEndC -\numbers[3]) {0};

  \node at (\xStartC +0.25, \yEndC -\numbers[4] -.2) {0};
  \node at (\xStartC +0.25, \yEndC -\numbers[5] -.2) {1};
  \node at (\xStartC +0.25, \yEndC -\numbers[6] -.2) {0};
  \node at (\xStartC +0.25, \yEndC -\numbers[7] -.2) {\vdots};
  \node at (\xStartC +0.25, \yStartC +\numbers[0]) {0};

  % Annotation
  \node at (\xStartC -\shiftMargin, \yStartA +\half) {\small 41};
  \draw[arrows=-angle 90, very thin]  (\xStartC -\shiftMargin, 2.5) -- (\xStartC -\shiftMargin, \yEndC -0.1);
  \draw[arrows=angle 90-, very thin]  (\xStartC -\shiftMargin, 0.5) -- (\xStartC -\shiftMargin, 2);

  \draw [decorate,decoration={brace,amplitude=2pt,mirror,raise=2pt}, very thin]
  (\xEndC +\shiftMargin -.1, 2.5 +.2) -- (\xEndC +\shiftMargin  -.1, \yEndC -0.1);

  \draw [decorate,decoration={brace,amplitude=2pt,mirror,raise=2pt}, very thin]
  (\xEndC +\shiftMargin -.1, 0.5) -- (\xEndC +\shiftMargin -.1, 2.4);


  \node [rectangle, text width= 5em,font=\footnotesize] at (5.6, 3.4)
    {\scriptsize $4 \times 1$ encoding vector representing diacritic \textarabic{◌َ}};

  \node [rectangle, text width= 5em,font=\footnotesize] at (5.6, 1.4)
    {\scriptsize $37 \times 1$ encoding vector representing letter \textarabic{ب}};

\end{tikzpicture}
};
    \node at (-12+0.3,-3){\small (a) \textit{One-Hot}};
    \node at (-7+0.3, -3){\small (b) \textit{Binary}};
    \node at (0-0.5,  -3){\small (c) \textit{Two-Hot}};
\end{tikzpicture}

  \caption{Three encoding schemes: \textit{One-hot} (a), \textit{binary} (b), and \textit{two-hot}
    (c). The example word \textarabic{مَرْحَبَا} consists of 5 letters and is used to illustrate the
    \textit{one-hot} and \textit{binary} encodings. One of its letters \textarabic{بَ} is selected as
    an example to illustrate the \textit{two-hot} encoding (c).}\label{fig:One-Binary-Encoding}
\end{figure*}
It was introduced in Sec.~\ref{sec:arab-poetry-text} that a poem meter, in particular Arabic poem,
is a phonetic pattern of vowels and consonants that is inferred from \textit{harakah} and
\textit{sukun} of the written text. It is therefore obvious that text should be fed to the network
at the character (not word) level. Characters are categorical predictors, and therefore character
encoding is necessary for feeding them to any form of Neural Networks (NN). Categorical variable
encoding has an impact on the neural network performance. (We elaborate on that upon discussing the
results in Sec.~\ref{sec:discussion}). E.g.,~\cite{Potdar2017ComparativeStudyCategoricalVariable} is
a comparative study for six encoding techniques. They have trained NN on the \textit{car evaluation}
dataset after encoding the seven ordered qualitative
features.~\cite{Agirrezabal2017ComparisonFeatureBasedNeural} shows that representations of data
learned from character-based neural models are more informative than the ones from hand-crafted
features.

In this research, we have used the two known encoding schemes \textit{one-hot} and \textit{binary},
in addition to the \textit{two-hot} that we introduced for more efficient encoding of the Arabic
letters. Before explaining these three encoding schemes, we need to make the distinction clear
among: letters, diacritics, characters (or symbols), and encoding vectors. In English language (and
even in Latin that has letters with diacritics, e.g., \textit{ê, é, è, ë, ē, ĕ, ě}), each letter is
considered a standalone character (or symbol) with a unique Unicode. Each of them is encoded to a
vector, whose length $n$ depends on the encoding scheme. Then, a word, or a verse, consisting of $p$
letters (or characters in this case) would be represented as $n \times p$ matrix. However, in Arabic
Language, diacritics are treated differently in the Unicode system. A diacritic is considered a
standalone character (symbol) with a unique Unicode (in contrast to Latin diacritics as just
explained). E.g., the Arabic letter \textarabic{بَ}, which is the letter \textarabic{ب} accompanied
with the diacritic \textarabic{◌َ}, is considered in Unicode system as two consecutive characters,
the character \textarabic{ب} followed by the character \textarabic{◌َ}, where each has its own
Unicode. Based on that, Arabic and English text are encoded using each of the three encoding methods
as follows.

\bigskip

\subsubsection{One-Hot encoding}\label{sec:one-hot-encoding}
In English, there are 26 letters, a white-space, and an apostrophe; hence, there are 28 final
characters. In \textit{one-hot} encoding each of the 28 characters will be represented by a vector
of length $n=28$ having a single one and 27 zeros; hence, this is a sparse encoding. In Arabic, we
will represent a combination of a letter and its diacritic together as a single encoding
vector. Since, from Sec.~\ref{sec:arabic-language}, there are 36 letters, 4 diacritics and a
white-space, and since a letter may or may not have a diacritic whereas the white-space cannot,
there is a total of $36 \times (4+1) + 1 = 181$ combinations. Hence, the encoding vector length is
$n=181$; each vector will have just a single one and 180 zeros.


\subsubsection{Binary Encoding}\label{sec:binary-encoding}
In binary encoding, an encoding vector of length $n$ contains a unique binary combination in
contrast to the sparse \textit{one-hot} encoding representation. Therefore, the encoding lengths of
English and Arabic are $\ceil{\log_2 28} = 5$ and $\ceil{\log_2 181} = 8$ respectively, which is a
huge reduction in dimensionality. However, this will be on the expense the challenge added to find
the best network architecture design that is capable of decoding this scheme
(Sec.~\ref{sec:discussion}).


\subsubsection{Two-Hot encoding}\label{sec:two-hot-encoding}
For Arabic language, where diacritics explode the length of the \textit{one-hot} encoding vector to
181, we introduce this new encoding. In this encoding, the 36 letters and the white-space on a hand
and the 4 diacritics on the other hand are encoded separately using two \textit{one-hot} encoding
vectors of lengths $n=37$ and $n=4$ respectively. The final \textit{two-hot} encoding of a
letter with a diacritic is the stacking of the two vectors to produce a final encoding vector of
length $n=37+4 = 41$. Clearly, a letter with no diacritic will have 4 zeros in the diacritic portion
of the encoding vector.

Figure~\ref{fig:One-Binary-Encoding} illustrates the three encoding schemes. The \textit{one-hot}
and \textit{binary} encoding of the whole 5-letter word \textarabic{مَرْحَبا} are illustrated as
$181 \times 5$ and $8 \times 5$ matrices respectively (Figures~\ref{fig:One-Binary-Encoding}-a,
\ref{fig:One-Binary-Encoding}-b). In Figure~\ref{fig:One-Binary-Encoding}-c only the second letter
of the word, \textarabic{بَ}, is taken an example to illustrate the \textit{two-hot} encoding. It is
obvious that the \textit{one-hot} is the most lengthy encoding; however, it is straightforward for
networks to decode since no two vectors share the same position of `1'. On the other extreme, the
\textit{binary} encoding is most economic one; however, networks may need careful design to decode
the pattern since vectors share many positions of `1's and `0's. Efficiently, the new designed
\textit{two-hot} encoding is almost 28\% of the size of \textit{one-hot} encoding.% ; however, still
